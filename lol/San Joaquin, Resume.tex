%!TEX TS-program = xelatex
%!TEX encoding = UTF-8 Unicode
% Awesome CV LaTeX Template
%
% This template has been downloaded from:
% https://github.com/posquit0/Awesome-CV
%
% Author:
% Claud D. Park <posquit0.bj@gmail.com>
% http://www.posquit0.com
%
% Template license:
% CC BY-SA 4.0 (https://creativecommons.org/licenses/by-sa/4.0/)
%


%%%%%%%%%%%%%%%%%%%%%%%%%%%%%%%%%%%%%%
%     Configuration
%%%%%%%%%%%%%%%%%%%%%%%%%%%%%%%%%%%%%%
%%% Themes: Awesome-CV
\documentclass[]{awesome-cv}
\usepackage{textcomp}
\hypersetup{%
  colorlinks=true,% hyperlinks will be coloured
  linkbordercolor=blue,% hyperlink border will be red
}
%%% Override a directory location for fonts(default: 'fonts/')
\fontdir[fonts/]

%%% Configure a directory location for sections
\newcommand*{\sectiondir}{resume/}

%%% Override color
% Awesome Colors: awesome-emerald, awesome-skyblue, awesome-red, awesome-pink, awesome-orange
%                 awesome-nephritis, awesome-concrete, awesome-darknight
%% Color for highlight
% Define your custom color if you don't like awesome colors
\colorlet{awesome}{awesome-red}
%\definecolor{awesome}{HTML}{CA63A8}
%% Colors for text
%\definecolor{darktext}{HTML}{414141}
%\definecolor{text}{HTML}{414141}
%\definecolor{graytext}{HTML}{414141}
%\definecolor{lighttext}{HTML}{414141}

%%% Override a separator for social informations in header(default: ' | ')
%\headersocialsep[\quad\textbar\quad]
\renewcommand*{\cventry}[5]{%
  \vspace{-2.0mm}
  \setlength\tabcolsep{0pt}
  \setlength{\extrarowheight}{0pt}
  \begin{tabular*}{\textwidth}{@{\extracolsep{\fill}} L{\textwidth - 4.5cm} R{4.5cm}}
    \ifempty{#2#3}
      {\ifempty{#1#4}{}{\entrypositionstyle{#1} & \entrydatestyle{#4} \\}}
      {\entrytitlestyle{#2} & \entrylocationstyle{#3} 
       \ifempty{#1#4}{}{\\\entrypositionstyle{#1} & \entrydatestyle{#4}}}
    \ifempty{#5}{}{\\\multicolumn{2}{L{\textwidth}}{\descriptionstyle{#5}}}
  \end{tabular*}%
  \par % <============================================================== missing in class
}



\begin{document}
    
%%%%%%%%%%%%%%%%%%%%%%%%%%%%%%%%%%%%%%
%     Profile
%%%%%%%%%%%%%%%%%%%%%%%%%%%%%%%%%%%%%%
\begin{center}
	\headerfirstnamestyle{\textbf{Ayrton}} \ \headerlastnamestyle{San Joaquin} \\
	\vspace{1mm}
	%\headerpositionstyle{Machine Learning Researcher | Writer} \\
	\headerpositionstyle{Researcher - Trustworthy Machine Learning (Privacy, Security) | Writer} \\
	\vspace{1mm}
	{\faEnvelope\ ayrton@yale-nus.edu.sg} | \iffalse {\faMobile\ +65 88147588} | \fi{\faMapMarker\ Copenhagen, Denmark} | {\faLinkedinSquare\ \href{https://www.linkedin.com/in/ajsanjoaquin/}{ajsanjoaquin}}| {\faLink\ \href{http://ajsanjoaquin.github.io/about}{ajsanjoaquin.github.io}}
\end{center}
%%%%%%%%%%%%%%%%%%%%%%%%%%%%%%%%%%%%%%
%     Education
%%%%%%%%%%%%%%%%%%%%%%%%%%%%%%%%%%%%%%
\vspace{-2mm}
\cvsection{Education}
\begin{cventries}
	\cventry
	{Bachelor of Science (Honors) in Data Science, Minor in Philosophy}
	{Yale-NUS College}
	{Singapore}
	{August 2018 - February 2023}
	{Awarded Scholarship to attend Full-time. Currently on exchange at K\o benhavn Universitet.}
\end{cventries}
\vspace{-4mm}
%%%%%%%%%%%%%%%%%%%%%%%%%%%%%%%%%%%%%%
%     Experience
%%%%%%%%%%%%%%%%%%%%%%%%%%%%%%%%%%%%%%
\cvsection{Experience}
\begin{cventries}
	\cventry
	{Teaching Assistant (Fall), Scholar (Summer)}
	{Machine Learning Safety Scholars Program, Center for AI Safety}
	{Palo Alto, United States}
	{June 2022 - December 2022}
	{\begin{cvitems}
		\item {Led a 5-person grading team handling 97 students worldwide, ranging from pre-university students to professionals.}
		\item {Led research on analyzing large language models' adaptability to new word definitions using few-shot learning.}
		\item {Received a grant of $\$4500$ to complete the inaugural program.}
	\end{cvitems}}
	\cventry
	{Undergraduate Researcher}
	{Data Privacy and Trustworthy Machine Learning Lab, NUS}
	{Singapore}
	{May 2021 - August 2021}
	{\begin{cvitems}
		\item {Pitched and led a project to analyze \href{https://arxiv.org/abs/2101.04898}{Unlearnable Data} as a data protection method.}
		\item {Collaborated with Google Brain on privacy attack research for my bachelor's thesis in a team across 4 timezones. Published as the youngest and only undergraduate co-author.}
	\end{cvitems}}
	\cventry
	{Deep Learning Engineer}
	{Arterys (Freelance)}
	{San Francisco, United States}
	{March 2020 - June 2020}
	{\begin{cvitems}
		\item {Created a COVID-19 Pneumonia classifier four days after pandemic declaration in collaboration with A.I. Singapore.}
		\item {Collaborated with Arterys to \href{https://marketplace.arterys.com/model/ayrtoncovidXR}{deploy the model in their platform} 
		for use by American hospitals and researchers. Model engineer in a team of 4 accross 3 timezones.}
	\end{cvitems}}
\end{cventries}

\vspace{-5mm}
\cvsection{Open-Source Projects \& Contributions}
\begin{cventries}
	\cventry
	{}
	{\href{https://github.com/ajsanjoaquin/Shapley_Valuation}{Equitable Valuation of Data Using Shapley Values}}
	{Data Protection}
	{}
	{• Implemented the training data valuation algorithm from \textit{What is your data worth? Equitable Valuation of Data (Ghorbani and Zou., 2019)}.}
	\vspace{-1mm}
	\cventry
	{}
	{\href{https://github.com/ajsanjoaquin/mPerturb}{Explaining Neural Networks with Meaningful Perturbations}}
	{Explainable AI}
	{}
	{• For explaining an image classifier's prediction, I implemented the algorithm described in \textit{Explanations of Black Boxes by Meaningful Perturbation (Fong, et. al., 2018)}.}
	
	\vspace{-1mm}
	\cventry
	{}
	{\href{https://github.com/ajsanjoaquin/COVID-19-Scanner}{COVID-19 Pneumonia Classifier for Diagnosis Triage}}
	{Medical Imaging}
	{}
	{• Trained a Resnet-34 Convolutional Neural Network (CNN) on \textasciitilde{} 26,000 images with Resampling to detect Pneumonia caused by COVID-19 on xray scans ultimately to triage patients for urgent diagnosis.}
	
	\vspace{-1mm}
	\cventry
	{}
	{\href{https://github.com/pulls?q=is\%3Apr+author\%3Aajsanjoaquin+archived\%3Afalse+is\%3Aclosed}{Miscellaneous}}
	{Machine Learning Community}
	{}
	{• Added new features for major machine learning projects including Pytorch, HuggingFace Transformers, and YOLOv4 (object detection model).}
\end{cventries}

\vspace{-3mm}
\cvsection{Publications}
\vspace{-3mm}
\begin{cvhonors}
	\cvhonor
	{San Joaquin, A., \textnormal{Haroen, A.} }
	{Understanding How Model Size Affects Few-shot Instruction Prompting}
	{\href{https://arxiv.org/abs/2212.01907}{ar$\chi$iv}}
	{December 2022}
	\cvhonor
	{San Joaquin, A., \textnormal{Skubacz, F.} }
	{Applying Multilingual Models to Question Answering (QA) }
	{\href{https://arxiv.org/abs/2212.01933}{ar$\chi$iv}}
	{December 2022}
	\cvhonor
	{\textnormal{Tramer, F., \dots ,} San Joaquin, A., \textnormal{ et.al.} }
	{Truth Serum: Poisoning Machine Learning Models to Reveal Their Secrets}
	{\href{https://dl.acm.org/doi/10.1145/3548606.3560554}{ACM CCS 2022}}
	{November 2022}
	\cvhonor
	{San Joaquin, A.}
	{\textit{Using Deep Learning to Detect Pneumonia caused by COVID-19}}
	{\href{https://towardsdatascience.com/using-deep-learning-to-detect-ncov-19-from-x-ray-images-1a89701d1acd}{Towards Data Science (Editor's Choice)}}
	{March 2020}
\end{cvhonors}

\vspace{-6mm}
\cvsection{Press}
\vspace{-3mm}
\begin{cvhonors}
	\cvhonor
	{\href{https://www.theregister.com/2022/04/12/machine_learning_poisoning}{Machine learning models leak personal info if training data is compromised}}
	{\textit{The Register}}
	{}
	{April 2022}
\end{cvhonors}

\vspace{-3mm}
\cvsection{Skills}
\vspace{-3mm}
\begin{cventries}
	\cventry
	{}
	{\def\arraystretch{1.15}{\begin{tabular}{ l l }
		Programming Languages:  & {\skill{ Python, Java, R}} \\
		Machine Learning in Python:  & {\skill{ Pytorch, Pytorch Lightning, NumPy, Sickit-Learn, Tensorflow, Keras, Jax}} \\
		Data Management:  & {\skill{ Pandas, SQL, MS Excel}} \\
		Application Deployment \& \\
		Version Control:  & {\skill{ Docker, Google Cloud, Git, Singularity}} \\
		\end{tabular}}}
	{}
	{}
	{}
\end{cventries}

\vspace{-20mm}
\end{document}