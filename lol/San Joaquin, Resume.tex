%!TEX TS-program = xelatex
%!TEX encoding = UTF-8 Unicode
% Awesome CV LaTeX Template
%
% This template has been downloaded from:
% https://github.com/posquit0/Awesome-CV
%
% Author:
% Claud D. Park <posquit0.bj@gmail.com>
% http://www.posquit0.com
%
% Template license:
% CC BY-SA 4.0 (https://creativecommons.org/licenses/by-sa/4.0/)
%


%%%%%%%%%%%%%%%%%%%%%%%%%%%%%%%%%%%%%%
%     Configuration
%%%%%%%%%%%%%%%%%%%%%%%%%%%%%%%%%%%%%%
%%% Themes: Awesome-CV
\documentclass[]{awesome-cv}
\usepackage{textcomp}
\hypersetup{%
  colorlinks=true,% hyperlinks will be coloured
  linkbordercolor=blue,% hyperlink border will be red
}
%%% Override a directory location for fonts(default: 'fonts/')
\fontdir[fonts/]

%%% Configure a directory location for sections
\newcommand*{\sectiondir}{resume/}

%%% Override color
% Awesome Colors: awesome-emerald, awesome-skyblue, awesome-red, awesome-pink, awesome-orange
%                 awesome-nephritis, awesome-concrete, awesome-darknight
%% Color for highlight
% Define your custom color if you don't like awesome colors
\colorlet{awesome}{awesome-red}
%\definecolor{awesome}{HTML}{CA63A8}
%% Colors for text
%\definecolor{darktext}{HTML}{414141}
%\definecolor{text}{HTML}{414141}
%\definecolor{graytext}{HTML}{414141}
%\definecolor{lighttext}{HTML}{414141}

%%% Override a separator for social informations in header(default: ' | ')
%\headersocialsep[\quad\textbar\quad]
\renewcommand*{\cventry}[5]{%
  \vspace{-2.0mm}
  \setlength\tabcolsep{0pt}
  \setlength{\extrarowheight}{0pt}
  \begin{tabular*}{\textwidth}{@{\extracolsep{\fill}} L{\textwidth - 4.5cm} R{4.5cm}}
    \ifempty{#2#3}
      {\ifempty{#1#4}{}{\entrypositionstyle{#1} & \entrydatestyle{#4} \\}}
      {\entrytitlestyle{#2} & \entrylocationstyle{#3} 
       \ifempty{#1#4}{}{\\\entrypositionstyle{#1} & \entrydatestyle{#4}}}
    \ifempty{#5}{}{\\\multicolumn{2}{L{\textwidth}}{\descriptionstyle{#5}}}
  \end{tabular*}%
  \par % <============================================================== missing in class
}



\begin{document}
    
%%%%%%%%%%%%%%%%%%%%%%%%%%%%%%%%%%%%%%
%     Profile
%%%%%%%%%%%%%%%%%%%%%%%%%%%%%%%%%%%%%%
\begin{center}
	\headerfirstnamestyle{\textbf{Ayrton}} \ \headerlastnamestyle{San Joaquin} \\
	{\faEnvelope\ ajsanjoaquin@gmail.com} | \iffalse {\faMobile\ +65 88147588} | \fi{\faMapMarker\ Singapore} | {\faLinkedinSquare\ \href{https://www.linkedin.com/in/ajsanjoaquin/}{ajsanjoaquin}}| {\faLink\ \href{http://ajsanjoaquin.github.io/}{ajsanjoaquin.github.io}}
\end{center}
%%%%%%%%%%%%%%%%%%%%%%%%%%%%%%%%%%%%%%
%     Education
%%%%%%%%%%%%%%%%%%%%%%%%%%%%%%%%%%%%%%
\cvsection{Education}
\begin{cventries}
	\cventry
	{Bachelor of Science (Honors) in Data Science, Minor in Philosophy}
	{Yale-NUS College}
	{Singapore}
	{August 2018 - February 2023}
	{Awarded Scholarship to attend Full-time}
\end{cventries}
\vspace{-2mm}
%%%%%%%%%%%%%%%%%%%%%%%%%%%%%%%%%%%%%%
%     Experience
%%%%%%%%%%%%%%%%%%%%%%%%%%%%%%%%%%%%%%
\cvsection{Experience}
\begin{cventries}
	\cventry
	{Undergraduate Researcher}
	{Data Protection and Trustworthy Machine Learning Lab, NUS}
	{Singapore}
	{May 2021 - August 2021}
	{\begin{cvitems}
		\item {Pitched and led a project to analyze \href{https://arxiv.org/abs/2101.04898}{Unlearnable Data} as a data protection method. Paper to be refined in a workshop.}
		\item {Collaborating with Google Brain on privacy attack research for my bachelor's thesis.}
	\end{cvitems}}
	\cventry
	{Deepfake Detection Research Intern}
	{NUS-Tsinghua Center For Extreme Search (NeXT++)}
	{Singapore}
	{May 2020 - August 2020}
	{\begin{cvitems}
		\item {Read and adapted various robustness strategies against adversarial noises (e.g. Adversarial Training, Randomized Smoothing)}
	\end{cvitems}}
	\cventry
	{Deep Learning Engineer}
	{Arterys (Freelance)}
	{San Francisco, United States}
	{March 2020 - June 2020}
	{\begin{cvitems}
		\item {Created a COVID-19 Pneumonia classifier four days after pandemic declaration in collaboration with A.I. Singapore.}
		\item {Contacted by Arterys, and \href{https://marketplace.arterys.com/model/ayrtoncovidXR}{Deployed model in the Arterys platform}, 
		alongside models from NVIDIA and Ping An Technology, for use by American hospitals and researchers.}
	\end{cvitems}}
\end{cventries}

\vspace{1mm}
\cvsection{Open-Source Projects \& Contributions}
\begin{cventries}
	\cventry
	{}
	{\href{https://github.com/ajsanjoaquin/twitter-bias-challenge}{Twitter Algorithmic Bias Challenge 2021}}
	{}
	{}
	{• Identified unintended sexualization of non-sexual images involving nudity by the \href{https://github.com/twitter-research/image-crop-analysis}{Twitter Image Cropper Algorithm}. Finished 9th out of 40 teams worldwide.}
	
	\vspace{-1mm}
	\cventry
	{}
	{\href{https://github.com/ajsanjoaquin/mPerturb}{Explaining Neural Networks with Meaningful Perturbations}}
	{}
	{}
	{• For explaining an image classifier's prediction, I implemented the algorithm described in \textit{Explanations of Black Boxes by Meaningful Perturbation (Fong, et. al., 2018)}.}
	
	\vspace{-1mm}
	\cventry
	{}
	{\href{https://github.com/ajsanjoaquin/COVID-19-Scanner}{COVID-19 Pneumonia Classifier for Diagnosis Triage}}
	{}
	{}
	{• Trained a Resnet-34 Convolutional Neural Network (CNN) on \textasciitilde{} 26,000 images with Resampling to detect Pneumonia caused by COVID-19 on xray scans ultimately to triage patients for urgent diagnosis.}
	
	\vspace{-1mm}
	\cventry
	{}
	{\href{https://github.com/pulls?q=is\%3Apr+author\%3Aajsanjoaquin+archived\%3Afalse+is\%3Aclosed}{Miscellaneous}}
	{}
	{}
	{• Contributed improvements to popular and massive projects including Pytorch (deep learning framework) and YOLOv4 (object detection model).}
\end{cventries}

\cvsection{Publications}
*No name indicates first or sole authorship.
\begin{cvhonors}
	\cvhonor
	{\textnormal{Tramer, F., \dots ,} San Joaquin, A., \textnormal{ et.al.} }
	{\href{https://arxiv.org/abs/2204.00032}{Truth Serum: Poisoning Machine Learning Models to Reveal Their Secrets}, \textit{To appear in ACM CCS 2022.}}
	{}
	{April 2022}
	\cvhonor
	{}
	{\textit{\href{https://towardsdatascience.com/using-deep-learning-to-detect-ncov-19-from-x-ray-images-1a89701d1acd}{Using Deep Learning to Detect Pneumonia caused by COVID-19} Towards Data Science}}
	{}
	{March 2020}
\end{cvhonors}

\cvsection{Press}
\begin{cvhonors}
	\cvhonor
	{\href{https://www.theregister.com/2022/04/12/machine_learning_poisoning}{Machine learning models leak personal info if training data is compromised}}
	{\textit{The Register}}
	{}
	{April 2022}
\end{cvhonors}

\cvsection{Skills}
\vspace{-3mm}
\begin{cventries}
	\cventry
	{}
	{\def\arraystretch{1.15}{\begin{tabular}{ l l }
		Programming Languages:  & {\skill{ Python, Java, R}} \\
		Machine Learning in Python:  & {\skill{ Pytorch, Pytorch Lightning, NumPy, Sickit-Learn, Tensorflow, Keras, Jax}} \\
		Data Management:  & {\skill{ Pandas, SQL, MS Excel}} \\
		Application Deployment \& \\
		Version Control:  & {\skill{ Docker, Google Cloud, Git, Singularity}} \\
		\end{tabular}}}
	{}
	{}
	{}
\end{cventries}

\vspace{-20mm}
\ 
\end{document}