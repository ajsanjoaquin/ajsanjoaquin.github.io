%!TEX TS-program = xelatex
%!TEX encoding = UTF-8 Unicode
% Awesome CV LaTeX Template
%
% This template has been downloaded from:
% https://github.com/posquit0/Awesome-CV
%
% Author:
% Claud D. Park <posquit0.bj@gmail.com>
% http://www.posquit0.com
%
% Template license:
% CC BY-SA 4.0 (https://creativecommons.org/licenses/by-sa/4.0/)
%


%%%%%%%%%%%%%%%%%%%%%%%%%%%%%%%%%%%%%%
%     Configuration
%%%%%%%%%%%%%%%%%%%%%%%%%%%%%%%%%%%%%%
%%% Themes: Awesome-CV
\documentclass[]{awesome-cv}
\usepackage{textcomp}
\hypersetup{%
  colorlinks=true,% hyperlinks will be coloured
  linkbordercolor=red,% hyperlink border will be red
}
%%% Override a directory location for fonts(default: 'fonts/')
\fontdir[fonts/]

%%% Configure a directory location for sections
\newcommand*{\sectiondir}{resume/}

%%% Override color
% Awesome Colors: awesome-emerald, awesome-skyblue, awesome-red, awesome-pink, awesome-orange
%                 awesome-nephritis, awesome-concrete, awesome-darknight
%% Color for highlight
% Define your custom color if you don't like awesome colors
\colorlet{awesome}{awesome-red}
%\definecolor{awesome}{HTML}{CA63A8}
%% Colors for text
%\definecolor{darktext}{HTML}{414141}
%\definecolor{text}{HTML}{414141}
%\definecolor{graytext}{HTML}{414141}
%\definecolor{lighttext}{HTML}{414141}

%%% Override a separator for social informations in header(default: ' | ')
%\headersocialsep[\quad\textbar\quad]
    \begin{document}
    
%%%%%%%%%%%%%%%%%%%%%%%%%%%%%%%%%%%%%%
%     Profile
%%%%%%%%%%%%%%%%%%%%%%%%%%%%%%%%%%%%%%
\begin{center}
	\headerfirstnamestyle{\textbf{Ayrton}} \ \headerlastnamestyle{San Joaquin} \\
	{\faEnvelope\ ayrton@u.yale-nus.edu.sg} | {\faMobile\ +65 881475 88} | {\faMapMarker\ Singapore} | {\faLinkedinSquare\ \href{https://www.linkedin.com/in/ajsanjoaquin/}{ajsanjoaquin}}| {\faLink\ \href{http://ajsanjoaquin.github.io/}{ajsanjoaquin.github.io}}
\end{center}
%%%%%%%%%%%%%%%%%%%%%%%%%%%%%%%%%%%%%%
%     Education
%%%%%%%%%%%%%%%%%%%%%%%%%%%%%%%%%%%%%%
\cvsection{Education}
\begin{cventries}
	\cventry
	{Bachelor of Science (Honors) in Data Science, Minor in Philosophy}
	{Yale-NUS College}
	{Singapore}
	{August 2018 – May 2022}
	{Awarded Scholarship to attend Full-time}
	\cventry
	{Certificate in Machine Learning (\href{https://www.coursera.org/account/accomplishments/verify/WFK75DQC9N5Q}{Credential ID: WFK75DQC9N5Q})}
	{Coursera
	\textcolor{white}{Masters , Doctorate, AI, Computer Vision}}
	{}
	{July 2019}
	{}
\end{cventries}
\vspace{-8mm}
%%%%%%%%%%%%%%%%%%%%%%%%%%%%%%%%%%%%%%
%     Experience
%%%%%%%%%%%%%%%%%%%%%%%%%%%%%%%%%%%%%%
\cvsection{Experience}
\begin{cventries}
	\cventry
	{Undergraduate Researcher}
	{Data Protection and Trustworthy Machine Learning Lab, NUS}
	{Singapore}
	{May 2021 - Present}
	{\begin{cvitems}
		\item {Pitched and led a project to analyze \href{https://arxiv.org/abs/2101.04898}{Unlearnable Data} as a data protection method. Paper to be refined in a workshop.}
		\item {Currently working on a membership inference attack for my bachelor's thesis and advised by Prof. Reza Shokri}
		\end{cvitems}}
	\cventry
	{Deepfake Detection Research Intern}
	{NUS-Tsinghua Center For Extreme Search (NeXT++)}
	{Singapore}
	{May 2020 – August 2020}
	{\begin{cvitems}
		\item {Processed $\sim$200,000 images from FaceForensics++ Dataset and trained various detector models (Based on EfficientNet and Xception Net) using a High Performance Computing Cluster}
		\item {Read and adapted various robustness strategies against adversarial noises (e.g. Adversarial Training, Randomized Smoothing)}
		\end{cvitems}}
	\cventry
	{Deep Learning Engineer (Volunteer)}
	{Arterys (Freelance)}
	{San Francisco, United States}
	{March 2020 – June 2020}
	{\begin{cvitems}
		\item {Created a COVID-19 Pneumonia classifier four days after pandemic declaration, and developed it on an IBM Power9 System provided by A.I. Singapore.}
		\item {Contacted by Arterys, and \href{https://marketplace.arterys.com/model/ayrtoncovidXR}{Deployed model in the Arterys platform}, 
		alongside models from NVIDIA and Ping An Technology, for use by American hospitals and researchers.}
		\end{cvitems}}
\end{cventries}

\vspace{-5mm}
\cvsection{Skills}
\vspace{-3mm}
\begin{cventries}
	\cventry
	{}
	{\def\arraystretch{1.15}{\begin{tabular}{ l l }
		Programming Languages:  & {\skill{ Python, Java, R}} \\
		Machine Learning in Python:  & {\skill{ Pytorch, Pytorch Lightning, NumPy, Sickit-Learn, Fastai}} \\
		Data Management:  & {\skill{ Pandas, SQL, MS Excel}} \\
		Application Deployment \& Version Control:  & {\skill{ Docker, Google Cloud, Git, Singularity}} \\
		\end{tabular}}}
	{}
	{}
	{}
\end{cventries}

\vspace{-10mm}
\cvsection{Open-Source Projects \& Contributions}
\begin{cventries}
	\cventry
	{}
	{\href{https://github.com/ajsanjoaquin/twitter-bias-challenge}{Twitter Algorithmic Bias Challenge 2021}}
	{}
	{}
	{• Identified unintended sexualization of non-sexual images involving nudity by the \href{https://github.com/twitter-research/image-crop-analysis}{Twitter Image Cropper Algorithm}. Finished 9th out of 40 teams worldwide.}
	
	\vspace{-1mm}
	\cventry
	{}
	{\href{https://github.com/ajsanjoaquin/mPerturb}{Explaining Neural Networks with Meaningful Perturbations}}
	{Pytorch, NumPy}
	{}
	{• Implemented the algorithm described in \textit{Explanations of Black Boxes by Meaningful Perturbation (Fong, et. al., 2018)}, which perturbs a given image by masking the regions essential for an Image classifier to make a prediction.}
	
	\vspace{-1mm}
	\cventry
	{}
	{\href{https://github.com/ajsanjoaquin/COVID-19-Scanner}{COVID-19 Pneumonia Classifier for Diagnosis Triage}}
	{Fastai, Pytorch, Pandas, Docker}
	{}
	{• Trained a Resnet-34 Convolutional Neural Network (CNN) on \textasciitilde{} 26,000 images with Resampling to detect Pneumonia caused by COVID-19 on xray scans ultimately to triage patients for urgent diagnosis. AUROC for labels "covid", "opacity", "nofinding" were at 99.97\%, 92.64\%, and 92.73\%, respectively.}

%	\vspace{-2mm}
%	\cventry
%	{}
%	{\href{https://github.com/pytorch/pytorch/commits?author=ajsanjoaquin}{Pytorch}}
%	{}
%	{}
%	{• Previously implemented fixes for the parallelization and graph modules. Currently working on minor bug fixes.}
	
	\vspace{-4mm}
\end{cventries}
\cvsection{Publications}
\begin{cvhonors}
	\cvhonor
	{\href{https://towardsdatascience.com/lets-keep-explainable-methods-practical-and-relevant-92e963ce3f64}{Let’s Keep Explainable Methods Practical and Relevant}}
	{\textit{Towards Data Science}}
	{}
	{February 2021}
	\cvhonor
	{\href{https://towardsdatascience.com/using-deep-learning-to-detect-ncov-19-from-x-ray-images-1a89701d1acd}{Using Deep Learning to Detect Pneumonia caused by COVID-19}}
	{\textit{Towards Data Science}}
	{}
	{March 2020}
	\cvhonor
	{\href{https://medium.com/swlh/three-things-i-learned-from-creating-fake-faces-using-ai-fc4c95282a37}{Three Things I learned from Creating Fake Faces Using A.I.}}
	{\textit{The Startup}}
	{}
	{January 2020}
\end{cvhonors}
\vspace{-20mm}
\ 
\end{document}